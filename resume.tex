%%%%%%%%%%%%%%%%%%%%%%%%%%%%%%%%%%%%%%%
% This is a modified ONE COLUMN version of
% the following template:
%
% Deedy - One Page Two Column Resume
% LaTeX Template
% Version 1.1 (30/4/2014)
%
% Original author:
% Debarghya Das (http://debarghyadas.com)
%
% Original repository:
% https://github.com/deedydas/Deedy-Resume
%
% IMPORTANT: THIS TEMPLATE NEEDS TO BE COMPILED WITH XeLaTeX
%
% This template uses several fonts not included with Windows/Linux by
% default. If you get compilation errors saying a font is missing, find the line
% on which the font is used and either change it to a font included with your
% operating system or comment the line out to use the default font.
%
%%%%%%%%%%%%%%%%%%%%%%%%%%%%%%%%%%%%%%
%
% TODO:
% 1. Integrate biber/bibtex for article citation under publications.
% 2. Figure out a smoother way for the document to flow onto the next page.
% 3. Add styling information for a "Projects/Hacks" section.
% 4. Add location/address information
% 5. Merge OpenFont and MacFonts as a single sty with options.
%
%%%%%%%%%%%%%%%%%%%%%%%%%%%%%%%%%%%%%%
%
% CHANGELOG:
% v1.1:
% 1. Fixed several compilation bugs with \renewcommand
% 2. Got Open-source fonts (Windows/Linux support)
% 3. Added Last Updated
% 4. Move Title styling into .sty
% 5. Commented .sty file.
%
%%%%%%%%%%%%%%%%%%%%%%%%%%%%%%%%%%%%%%%
%
% Known Issues:
% 1. Overflows onto second page if any column's contents are more than the
% vertical limit
% 2. Hacky space on the first bullet point on the second column.
%
%%%%%%%%%%%%%%%%%%%%%%%%%%%%%%%%%%%%%%
    \documentclass[]{deedy-resume-openfont}

    \begin{document}
    
%%%%%%%%%%%%%%%%%%%%%%%%%%%%%%%%%%%%%%
%
%     Profile
%
%%%%%%%%%%%%%%%%%%%%%%%%%%%%%%%%%%%%%%
\namesection{Austin}{Bodzas}{abb6499@rit.edu | (908) 910-0098 | Rochester, NY}
%%%%%%%%%%%%%%%%%%%%%%%%%%%%%%%%%%%%%%
%
%     Education
%
%%%%%%%%%%%%%%%%%%%%%%%%%%%%%%%%%%%%%%
\section{Objective}
\raggedright
\runsubsection{Full-time}
To obtain a position as a Software Engineer starting in the Summer of 2019.

\descript{}\hfill \location{}\\
\section{Education}
\raggedright

\runsubsection{Rochester Institute of Technology}\descript{| BS Computer Science}\hfill \location{Rochester, NY | May 2019}\\
GPA: 3.7\\
\sectionsep
  

\runsubsection{Raritan Valley Community College}\descript{| AS Computer Science}\hfill \location{Branchburg, NJ | Dec 2016}\\
GPA: 3.75\\
\sectionsep
%%%%%%%%%%%%%%%%%%%%%%%%%%%%%%%%%%%%%%
%
%     Experience
%
%%%%%%%%%%%%%%%%%%%%%%%%%%%%%%%%%%%%%%
\section{Experience}
\runsubsection{L3 Global Communications Solutions}\hfill\descript{Software Engineer Co-Op}\\
\hfill \location{Victor, NY | Jan 2017 – Aug 2017}\\
\worksubheader{VSAT Ground Stations}
\begin{tightemize}
	\item Developed embedded software in C for AVR devices
    \item Gained experience in C++ development for embedded Linux
    \item Worked in-depth with serial communication
    \item Created driver interfaces to ancillary hardware
\end{tightemize}
\sectionsep

\runsubsection{Johns Hopkins Applied Physics Laboratory}\hfill\descript{Software Engineer Co-Op}\\
\hfill \location{Laurel, MD | Jan 2018 – Present}\\
	\worksubheader{NASA DART - Double Asteroid Redeflection Test Mission}
	\begin{tightemize}
		\item Wrote flight software utilizing NASA Core Flight Executive.
		\item Adapted Ball Aerospace Cosmos to work with DART's command and data handling system.
		\item Leveraged code reuse by porting over software from Parker Solar Probe for DART.
		\item Worked on DART's software testbed developing in C++, also utilizing NASA cFE.
	%%Memeber of the Flight Software team.  Responsible for adapting Ball Aerospace Cosmos ground software (GSW) to DART's command and telemetry system. 
	%%Gained experience with NASA Core Flight Executive by supporting DART applications.  Leveraged code reuse by porting over software from a previous APL mission, Parker Solar Probe.
	%%In the process discovered a bug in Parker Solar Probe's code that was fixed and set to launch in August.
	\end{tightemize}

\worksubheader{Software in the Loop Environment for Testing Flight Software (SWIL)}\\
\begin{tightemize}
	\item Setup Docker and Docker Compose environment to bring up flight, testbed, and ground software in one network on one machine.
	\item Created adapters to relay SpaceWire traffic over UDP for testbed and flight software.
	\item Software in the Loop (SWIL) allows for parallelized testing within a CI/CD system. Allows for portable developer environment across OSes.
	%Used in development and testing of flight software. Utilizes containerization of flight software (FSW), testbed software (TBSW), and ground software (GSW).  Directly responsible for containerizing Ball Aerospace Cosmos GSW and TBSW. Developed UDP adapter of SpaceWire packets to allow for easy Docker Compose networking between FSW and TBSW.
	%Docker devops to run fsw development in a reproducible environment and the same setup in Bamboo CI/CD for parallelized automated testing.
\end{tightemize}
	
\worksubheader{Deep Learning for Space – Internal Research and Development (IRAD)}\\
\begin{tightemize}
	\item Ported over DART flight software to run on an arm64 Jetson TX2 running Ubuntu
	\item Adapted DART testbed and ground software for the IRAD to allow for telemetry and command testing.
	\item Developed NASA cFE application to integrate with an image classifier and telemeter classifications
	\item Implemented ground softare to process incoming data and display on an OpenLayer map using web technology.
\end{tightemize}
	%Ported over DART FSW to run on an arm64 Jetson TX2.  Adapted DART TBSW and FSW to work with this IRAD as well for reuse.  Developing a NASA cFE application to integrate with a trained image classifier and transfer files from space. Developed ground software to process downlinked files and visualize classifications using OpenLayer maps.

\worksubheader{JHU APL Explorer Award - DART FSW Team}\\
	Awarded for the DART's team in utilizing technologies new to the laboratory such as Docker.

\worksubheader{Dockercon}\\
	Attended the Dockercon 2018 San Francisco conference. Work on SWIL was presented to main audience demonstrating the use for containerization in the public space sector.
	
\sectionsep
%%%%%%%%%%%%%%%%%%%%%%%%%%%%%%%%%%%%%%
%
%     Skills
%
%%%%%%%%%%%%%%%%%%%%%%%%%%%%%%%%%%%%%%
\section{Skills}
\raggedright
\begin{tabular}{ l l }
	\descript{Programming Languages} & {\location{C, C++, Python, Java, Bash, \LaTeX{}}}   \\
	\descript{Software}              & {\location{Git, Docker, NASA cFE, Ball Aerospace Cosmos, Bamboo Automated Testing}} \\
\end{tabular}
\sectionsep
%%%%%%%%%%%%%%%%%%%%%%%%%%%%%%%%%%%%%%
%
%     Projects
%
%%%%%%%%%%%%%%%%%%%%%%%%%%%%%%%%%%%%%%
\section{Involvement}
\raggedright

\runsubsection{{RIT Space Exploration (SPEX) Member}}\hfill\descript{Jan 2016 - Present}
\begin{tightemize}
	\item Current role as SPEX Technical Coordinator
	\item Avionics team lead for Cubesat Launch Initiative proposal
	\item Wrote software for SPEX's Hight Altitude Balloon 3 integraing with various sensors
	\item Explored potential cubesat payload for testing new memory technologies
\end{tightemize}
\runsubsection{{RIT SPEX Admin - Technical Coordinator}}\hfill\descript{June 2018 - Present}
\begin{tightemize}
	\item Interact with all the project teams in a managerial role
	\item Ensure teams have direction needed to achieve objectives
	\item Act as technical bridge between admins and project members
\end{tightemize}
	
\sectionsep
%%%%%%%%%%%%%%%%%%%%%%%%%%%%%%%%%%%%%%
%
%     Awards
%
%%%%%%%%%%%%%%%%%%%%%%%%%%%%%%%%%%%%%%
\end{document}