%%%%%%%%%%%%%%%%%%%%%%%%%%%%%%%%%%%%%%%
% This is a modified ONE COLUMN version of
% the following template:
%
% Deedy - One Page Two Column Resume
% LaTeX Template
% Version 1.1 (30/4/2014)
%
% Original author:
% Debarghya Das (http://debarghyadas.com)
%
% Original repository:
% https://github.com/deedydas/Deedy-Resume
%
% IMPORTANT: THIS TEMPLATE NEEDS TO BE COMPILED WITH XeLaTeX
%
% This template uses several fonts not included with Windows/Linux by
% default. If you get compilation errors saying a font is missing, find the line
% on which the font is used and either change it to a font included with your
% operating system or comment the line out to use the default font.
%
%%%%%%%%%%%%%%%%%%%%%%%%%%%%%%%%%%%%%%
%
% TODO:
% 1. Integrate biber/bibtex for article citation under publications.
% 2. Figure out a smoother way for the document to flow onto the next page.
% 3. Add styling information for a "Projects/Hacks" section.
% 4. Add location/address information
% 5. Merge OpenFont and MacFonts as a single sty with options.
%
%%%%%%%%%%%%%%%%%%%%%%%%%%%%%%%%%%%%%%
%
% CHANGELOG:
% v1.1:
% 1. Fixed several compilation bugs with \renewcommand
% 2. Got Open-source fonts (Windows/Linux support)
% 3. Added Last Updated
% 4. Move Title styling into .sty
% 5. Commented .sty file.
%
%%%%%%%%%%%%%%%%%%%%%%%%%%%%%%%%%%%%%%%
%
% Known Issues:
% 1. Overflows onto second page if any column's contents are more than the
% vertical limit
% 2. Hacky space on the first bullet point on the second column.
%
%%%%%%%%%%%%%%%%%%%%%%%%%%%%%%%%%%%%%%
    \documentclass[]{deedy-resume-openfont}

    \begin{document}
%%%%%%%%%%%%%%%%%%%%%%%%%%%%%%%%%%%%%%
%
%     Profile
%
%%%%%%%%%%%%%%%%%%%%%%%%%%%%%%%%%%%%%%
\namesection{Austin}{Bodzas}{abb6499@rit.edu | 9089100098 | Rochester, New York | github.com/Beaudidly}
%%%%%%%%%%%%%%%%%%%%%%%%%%%%%%%%%%%%%%
%
%     Education
%
%%%%%%%%%%%%%%%%%%%%%%%%%%%%%%%%%%%%%%
\section{Education}
\raggedright

\runsubsection{Raritan Valley Community College}\descript{| AS Computer Science}\hfill \location{Branchburg, NJ | December 2015}\\
GPA: 3.75\\

\runsubsection{Rochester Institute of Technology}\descript{| BS Computer Science}\hfill \location{Rochester, NY | December 2018}\\
GPA: 3.7\\

\sectionsep
%%%%%%%%%%%%%%%%%%%%%%%%%%%%%%%%%%%%%%
%
%     Experience
%
%%%%%%%%%%%%%%%%%%%%%%%%%%%%%%%%%%%%%%
\section{Experience}
\runsubsection{L3 Technologies GCS}\descript{| Software Engineer Co-Op}\hfill \location{Victor, NY | January 2017 – August 2017}
\begin{tightemize}
	\item Developed embedded software in C for AVR devices
    \item Gained experience in C++ development for embedded Linux
    \item Worked heavily with serial communication
    \item Created driver interfaces to ancillary hardware
\end{tightemize}
\sectionsep
%%%%%%%%%%%%%%%%%%%%%%%%%%%%%%%%%%%%%%
%
%     Skills
%
%%%%%%%%%%%%%%%%%%%%%%%%%%%%%%%%%%%%%%
\section{Skills}
\raggedright
\begin{tabular}{ l l }
    \descript{Languages} & {\location{C++, C, Python, Java, \LaTeX{}, JavaScript}} \\
    \descript{Operating Systems} & {\location{Linux (Arch, Ubuntu), OSX, Windows}} \\
    \descript{Software} & {\location{SVN, Git, CMake, GNU coreutils, Make, FreeRTOS}} \\
    \descript{Misc} & {\location{SPI, I\textsuperscript{2}C, Serial, Oscilliscope operation}}
\end{tabular}
\sectionsep
%%%%%%%%%%%%%%%%%%%%%%%%%%%%%%%%%%%%%%
%
%     Projects
%
%%%%%%%%%%%%%%%%%%%%%%%%%%%%%%%%%%%%%%
\section{Projects}
\raggedright

\runsubsection{\large{team lead for RIT space exploration CubeSat avionics: Fall 2016 - Spring 2017}}
\descript{}\hfill \location{}\\
Working as Team Lead for the Avionics subsystem for the RIT Space Exploration student faculty
research group (SPEX). Responsible for keeping the team on track for planning, designing, and developing
software thatruns the main computer onboard the cube satellite. The Avionic
team's task was to research the hardware and software necessary to control
a Cube Satellite in orbit and complete a scientific mission. Gained knowledge
on the increased challenge of computation in a space environment along with
technologies such as Real Time Operating Systems.
\bigbreak
\runsubsection{\large{High Altitude Balloon (HAB) Telemetry}}
\descript{}\hfill \location{}\\
Designed and developed telemetry software for SPEX HAB’s December 2016 launch. Responsibilities of
the software was to log atmospheric pressure, temperature, acceleration,rotation, and magnetic data to an
SD card. The project was written in C++, connecting to various sensors over serial bus protocols, SPI and
I\textsuperscript{2}C. Analysis of this recovered telemetry data proved critical to diagnosing a mid-flight mechanical failure.
\sectionsep

%%%
% Current Projects
%%%
\raggedright

\runsubsection{\large{IREC}}
\descript{}\hfill \location{}\\
IREC stuff

\bigbreak
\runsubsection{\large{microHab}}
\descript{}\hfill \location{}\\
MicroHab stuff
\ 
\end{document}
